After some hyperparameter tuning, our baseline U-Net model achieved IoU scores
between 0.7-0.9 for structures of interest (i.e. myocardium, blood
pool and atrium). Best baseline results were achieved with frozen encoder
weights, and convergence criteria was reached after 50 epochs.  \newline

Hyperparameters were unchanged when testing the different augmentation methods,
as the objective was not to achieve the highest possible segmentation accuracy -
but demonstrate if the augmentation techniques had a significant effect on
segmentation accuracy. For the real dataset, convergence was reached after 30
epochs and IoU scores ranged from 0.8-0.9. For the generated dataset,
convergence was reached after 80 epochs and IoU scores ranged from 0.7-0.8. A
table of results is provided below.
\newline

\begin{table}[h]
    \centering
    \begin{tabular}{|c|c|}
    \hline
    \textbf{Augmentation}         & \textbf{mIoU} \\ \hline
    None                          & 0.6           \\ \hline
    Real Elastic Deformation      & 0.9           \\ \hline
    Generated Elastic Deformation & 0.8           \\ \hline
    \end{tabular}
\end{table}

This gives strong evidence that using a pre-trained generator gives superior
data augmentation and segmentation accuracy than applying data augmentation
simply to the training images. \newline