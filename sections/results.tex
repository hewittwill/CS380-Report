After hyperparameter tuning, our baseline U-Net model achieved IoU scores
between 0.69-0.78 for structures of interest (LV Myocardium, LV Cavity and Left
Atrium - LA). Best baseline results were achieved with frozen encoder weights,
and convergence was marked at 50 epochs.  \newline

Hyperparameters were unchanged when testing the different augmentation methods,
as the objective was not to achieve the highest possible segmentation accuracy -
but demonstrate if the augmentation techniques had a significant effect on
segmentation accuracy. Both the real and generated models were trained for 50
epochs. For the real dataset IoU scores ranged from 0.72-0.82. For the generated
dataset IoU scores ranged from 0.72-0.80. A table of results is provided below.
\newline

\begin{table}[H]
    \centering
    \begin{tabular}{c|c|l|l|}
        \cline{2-4}
        \multicolumn{1}{l|}{}                       & \multicolumn{3}{c|}{\textbf{Intersection over Union}}                     \\ \hline
        \multicolumn{1}{|l|}{\textbf{Augmentation}} & LA                   & \multicolumn{1}{c|}{LVM} & \multicolumn{1}{c|}{LV} \\ \hline
        \multicolumn{1}{|c|}{None}                  & 0.69 ± 0.01          & 0.68 ± 0.01              & 0.78 ± 0.01             \\ \hline
        \multicolumn{1}{|c|}{Real}                  & 0.74 ± 0.02          & \textbf{0.72 ± 0.01}     & \textbf{0.82 ± 0.01}    \\ \hline
        \multicolumn{1}{|c|}{Generated}             & \textbf{0.75 ± 0.01} & \textbf{0.72 ± 0.01}     & 0.80 ± 0.01             \\ \hline
    \end{tabular}
\end{table}

This shows that there was an impact applying augmentation in general (both real
and generated augmentation), but does not appear to give a significantly
different effect between the real and generated forms of data augmentation.
\newline