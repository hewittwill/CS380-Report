Two-dimensional echocardiography plays a critical role in the clinical routine
as a low-cost, non-radiating and non-invasive method of assessing cardiac
structure and function. The analysis of echocardiography generally involves the
extraction of clinical measures, generally by manual or only semi-automated
techniques. \newline

An example of one such clinical measure is left ventricular ejection fraction
(LVEF), a ratio between the volume of the heart at End Systole (ES) and End
Diastole (ED). Heart volume measurements rely on the accurate segmentation of
the left ventricular myocardium, from the left ventricular cavity, at both ES
and ED. In the present typical clinical routine, these segmentations are made
manually by clinicians leading to poor accuracy and a lack of reproducibility of
measurements between readers. \newline

Convolutional Neural Networks (CNNs) have played an increasingly important role
in the field of medical image processing, their ability to extract complex
features from very noisy data has 

In this project, we first establish a baseline for the segmentation of the LV using a
encoder-decoder convolutional neural network. \newline

We then demonstrate that a Conditional Generative Adversarial Network (C-GAN)
can be used to generate photorealistic ultrasound images, from a ground truth
segmentation map. \newline

Then the C-GAN based augmentation approach is compared to image-processing
dataset augmentations, to establish if a significant effect on performance can
be found.